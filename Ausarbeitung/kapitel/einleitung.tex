\chapter{Einleitung}

Neuronale Netzwerke bilden eine Unterkategorie des Machine-Learning und erlauben
Auswertungen von Eingaben auf Basis von zuvor angelernten, empirischen
Ergebnissen.
Ein neuronales Netzwerk bildet ein System aus Neuronen ab, welche Schichtweise
verbunden sind einen unidirektionalen Datenfluss erzeugen.
Dieses System besteht üblicherweise aus einer Eingabeschicht, einer
Ausgabeschicht und dazwischen beliebig viele \emph{versteckte} Schichten,
welche die eigentliche Arbeit des Netzwerks verrichten.
Die Anzahl der Neuronen in den Eingabe- und Ausgabeschichten ist intuitiv
wählbar. Die Größe der Eingabeschicht wird häufig durch die Anzahl der
möglichen Eingaben bestimmt, die Größe der Ausgabeschicht durch die Anzahl
der möglichen Ergebnisse.
Die Größe und Anzahl der dazwischen liegenden Schichten hingegen muss je
nach Anforderung und Gegebenheiten individuell ermittelt werden.
Je größer das Netzwerk desto höher sind die Anforderungen an die benötigte
Hardware um dieses zu betreiben. Bei schwächerer Hardware oder
Einschränkungen bezüglich der Energieversorgung können kleinere Netzwerke
eingesetzt werden, wenn auch häufig mit geringerer Genauigkeit verglichen
mit einem größeren Netzwerk.


\section{Motivation}

Die Größe eines Netzwerks kann beim Entwurf dessen direkt beeinflusst werden.
In Anwendungsgebieten, bei denen der Fokus auf geringen Hardwareanforderungen
liegt, stoßen schnell das Problem der schwindenden Genauigkeit.
Auf IoT-Geräten oder mobilen Platformen finden klassische neuronale Netzwerke
daher nur eingeschränkt Nutzen.
2016 veröffentlichten M. Courbariaux und Y. Bengio \cite{Courbariaux} eine 
wissenschaftliche Arbeit und stellen dort das
\emph{binarisierte neuronale Netzwerk} (kurz BNN) vor.
Dieses könne im Vergleich zu einem klassischen neuronalen Netzwerk eine
theoretische Geschwindigkeitssteigerung auf das 32-fache erreichen.
Die Genauigkeit des BNN liege jedoch nur knapp unter derer klassischer Netzwerke.
Das BNN ermöglicht dank dieser Eigenschaften den Einsatz von neuronalen Netzen
auf vergleichsweise schwacher Hardware und verspricht zugleich
nur geringe Genauigkeitseinbüße.

In dieser Ausarbeitung wird die allgemeine Funktionsweise von neuronalen
Netzwerken erläutert und anschließend der Entwurf eines binarisiertes
Netzwerk mit Fokus auf Handschriftenerkennung auf Basis des MNIST Datensatzes
dokumentiert.
Dabei werden grundlegende Überlegungen, das Vorgehen, Herausforderungen
sowie dazu erarbeitete Lösungen vorgestellt.