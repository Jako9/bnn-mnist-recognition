\chapter{Einleitung}

Neuronale Netzwerke bilden eine Unterkategorie des Machine-Learning und erlauben
Auswertungen von Eingaben auf Basis von zuvor angelernten, empirischen
Ergebnissen.
Ein neuronales Netzwerk bildet ein System aus Neuronen ab, welche Schichtweise
verbunden sind einen unidirektionalen Datenfluss erzeugen.
Dieses System besteht üblicherweise aus einer Eingabeschicht, einer
Ausgabeschicht und dazwischen beliebig viele \emph{versteckte} Schichten,
welche die eigentliche Arbeit des Netzwerks verrichten.
Die Anzahl der Neuronen in den Eingabe- und Ausgabeschichten ist intuitiv
wählbar. Die Größe der Eingabeschicht wird häufig durch die Anzahl der
möglichen Eingaben bestimmt, die Größe der Ausgabeschicht durch die Anzahl
der möglichen Ergebnisse.
Die Größe und Anzahl der dazwischen liegenden Schichten hingegen muss je
nach Anforderung und Gegebenheiten individuell ermittelt werden.
Je größer das Netzwerk desto höher sind die Anforderungen an die benötigte
Hardware um dieses zu betreiben. Bei schwächerer Hardware oder
Einschränkungen bezüglich der Energieversorgung können kleinere Netzwerke
eingesetzt werden, wenn auch häufig mit geringerer Genauigkeit verglichen
mit einem größeren Netzwerk.


\section{Motivation}

Die Größe eines Netzwerks kann beim Entwurf dessen direkt beeinflusst werden.
In Anwendungsgebieten, bei denen der Fokus auf geringen Hardwareanforderungen
liegt, stoßen schnell das Problem der schwindenden Genauigkeit.
Auf IoT-Geräten oder mobilen Platformen finden klassische neuronale Netzwerke
daher nur eingeschränkt Nutzen.
In 2016 wurde von Courbariaux (TODO: \dots) eine Arbeit veröffentlich,
welche das Binarisierte neuronale Netzwerk (kurz \emph{BNN}) vorstellt,
einem neuronalen Netzwerk welches bis zu 32-mal schneller arbeiten kann
als ein klassisches neuronales Netzwerk.
Die Genauigkeit des BNN könne zudem mit der Genauigkeit von jenen klassischen
Netzwerken mithalten, wenn auch mit einer leichten Verschlechterung.
Das BNN ermöglicht dank seiner Eigenschaften den Einsatz von
neuronalen Netzen auf vergleichsweise schwacher Hardware und verspricht zugleich
nur geringe Genauigkeitseinbüße.

In dieser Ausarbeitung wird die allgemeine Funktionsweise von neuronalen
Netzwerken erläutert und anschließend der Entwurfs eines binarisiertes
Netzwerk dokumentiert.
Dabei werden grundlegende Überlegungen, das Vorgehen, auftretende Probleme
sowie dazu ermittelte Lösungen vorgestellt.